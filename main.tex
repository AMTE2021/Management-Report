\documentclass{article}
\usepackage[utf8]{inputenc}

\usepackage{a4wide}


\title{AMTE21 - Management report}
\author{Patrick Diehl \\ Irina Demeshko \\ Zhara Khatami \\ Steven R. Brandt \\ Parsa Amini}
\date{\today}

\begin{document}

\maketitle

%%%%%%%%%%%%%%%%%%%%%%%%%%%%%
\section{Key indicators}
%%%%%%%%%%%%%%%%%%%%%%%%%%%%%
The first submission deadline was May 7th with an extension to May 14th. We received three full papers at the first extension. Note that the workshop does not accept short papers. The deadline was extended to June 14th, and we received five submissions at the final deadline. In total, 8 papers were submitted on June 14th and x papers were accepted (x\%). The workshop was scheduled at X, and we had an average attendance of x persons. The keynote was given by Thomas Fahringer and x persons attended the keynote. The invited talks was given by x and x persons attended. The workshops closed with a panel discussion and x persons attended the panel discussion. More details are available on the workshop's webpage\footnote{https://amte2021.stellar-group.org/}.

%%%%%%%%%%%%%%%%%%%%%%%%%%%%%
\section{Committees}
%%%%%%%%%%%%%%%%%%%%%%%%%%%%%

%%%%%%%%%%%%%%%%%%%%%%%%%%%%%
\subsection{Organizing committee}
\label{sec:committee}
%%%%%%%%%%%%%%%%%%%%%%%%%%%%%

\begin{itemize}
    \item Irina Demeshko, Los Alamos National Laboratory, USA
    \item Patrick Diehl, Center for Computation \& Technology at Louisiana State University, USA
    \item Zahra Khatami, NVIDIA, USA
    \item Steven R. Brandt, Center for Computation \& Technology at Louisiana State University, USA
    \item Parsa Amini, Center for Computation \& Technology at Louisiana State University, USA
\end{itemize}


%%%%%%%%%%%%%%%%%%%%%%%%%%%%%
\subsection{Program committee}
%%%%%%%%%%%%%%%%%%%%%%%%%%%%%
Note that we tried to have a good mixture between European-based and American-based program committee members.

\begin{itemize}
    \item Metin H. Aktulga, Michigan State University, USA
    \item Bryce Adelstein Lelbach, NVIDIA, USA
    \item John Biddiscombe, Swiss National Supercomputing Centre, Switzerland
    \item Gregor Daiss, University of Stuttgart, Germany
    \item Vassilios Dimakopoulos, University of Ioannina, Greece
    \item Patricia Grubel, Los Alamos National Laboratory, USA
    \item Jeff Hammond, NVIDIA, USA
    \item Adrian Lemoine, AMD, USA
    \item Roman Lakymchuk, Fraunhofer ITWM, Germany
    \item Thomas Heller, Exasol, Germany
    \item Kevin Huck, University of Oregon, USA
    \item Daisy Hollman, Sandia National Laboratories, USA
    \item Laxmikant (Sanjay) V. Kale, University of Illinois at Urbana-Champaign, USA
    \item Hartmut Kaiser, Center for Computation \& Technology at Louisiana State University, USA
    \item Erwin Laure, Max Planck Computing \& Data Facility, Germany
    \item Andrew Lumsdaine, Northwest Institute for Advanced Computing, USA
    \item Pat McCormick, Los Alamos National Laboratory, USA
    \item Dirk Pleiter, Jülich Supercomputing Centre, Germany
    \item Galen Shipman, Los Alamos National Laboratory, USA
    \item Mikael Simberg, Swiss National Supercomputing Centre, Switzerland
    \item Sean Treichler, NVIDIA, USA
    \item Didem Unat, Koç University, Turkey

\end{itemize}

%%%%%%%%%%%%%%%%%%%%%%%%%%%%%
\subsection{Reviewers}
%%%%%%%%%%%%%%%%%%%%%%%%%%%%%
In addition to the organizers these program committee members reviewed at least one paper:
\begin{itemize}
\item Hartmut Kaiser, Center for Computation \& Technology at Louisiana State University, USA
\item Didem Unat, Koç University, Turkey
\item Dirk Pleiter, Jülich Supercomputing Centre, Germany
\item Roman Lakymchuk, Fraunhofer ITWM, Germany
\item Vassilios Dimakopoulos, University of Ioannina, Greece
\item Thomas Heller, Exasol, Germany
\item Jeff Hammond, NVIDIA
\item Dirk Pleiter, Jülich Supercomputing Centre, Germany
\item Hartmut Kaiser, Center for Computation \& Technology at Louisiana State University, USA
\item Erwin Laure, Max Planck Computing \& Data Facility, Germany
\item Gregor Daiss, University of Stuttgart, Germany
\item Vassilios Dimakopoulos, University of Ioannina, Greece
\item Patricia Grubel, Los Alamos National Laboratory, USA
\item Jeff Hammond, NVIDIA, USA
\item Adrian Lemoine, AMD, USA
\item Roman Lakymchuk, Fraunhofer ITWM, Germany
\end{itemize}
The reviewers were assigned to the papers based on their expertise and avoiding any conflict of interest. 


%%%%%%%%%%%%%%%%%%%%%%%%%%%%%
\section{Review process management}
%%%%%%%%%%%%%%%%%%%%%%%%%%%%%
Each paper was reviewed by at least four independent reviewers selected from the initial program committee using EasyChair. Note that all reviewers are listed in Section~\ref{sec:committee}. The deadline for the reviewers was set to June 25th. Part of the organization committee meet virtually on June 26th to discuss the reviewers rating on boarder line papers and made the final decision of acceptance. The final notification was sent to the authors using EasyChair on June 30th. The revised paper was requested by August 1st combined with the \LaTeX sources of the paper. All papers were uploaded to iThenticate for plagiarism check and the report was provided to the submitting author once the notifications were sent. The authors were asked to address the highlighted issues before the submission of the camera ready version. 


\section{Porgramm}

\subsection{Talks}

\begin{itemize}
    \item 9:00am - 9:05 am, Opening Remarks
    \item 9:05am - 10:05 am, Keynote Talk (Thomas Fahringer, UIBK)
    \item10:05am - 10:20am, Morning Break
    \item 10:20am - 10:50am Selected paper talk
    \item 10:50am - 11:20am, Selected paper talk
    \item 11:20am - 11:50am, Selected paper talk
    \item 11:50am - 12:20pm, Selected paper talk
    \item 12:20pm - 1:30pm, Lunch Break
    \item 1:30pm - 2:30pm, Invited talk (Daisy S. Hollman, Sandia National Laboratory, USA)
    \item 2:30pm - 3:00pm, Selected paper talk
    \item 3:00pm - 3:30pm, Selected paper talk
    \item 3:30pm -3:45 pm Afternoon break
    \item 3:45pm - 4:45pm, Panel
\end{itemize}


\subsection{Panel}

Moderator: Irina Demeshko, Los Alamos National Laboratory, USA\\

Panelists:

\begin{itemize}
    \item Hartmut Kaiser, Center for Computation \& Technology at Louisiana State University, USA
    \item  Laxmikant (Sanjay) Kale, Department of Computer Science, University of Illinois at Urbana-Champaign, USA
    \item  Martin Berzins, Scientific Computing and Imaging Institute, University of Utah, USA
    \item  Mike Bauer, NVIDIA, USA
    \item Thomas Fahringer, University of Innsbruck, Austria
\end{itemize}

Panel's chosen questions:

\begin{itemize}
    \item 
    Why should one choose to use AMT models?
    \item
    What kinds of applications will benefit from using AMT models and programming systems?
    \item
    For what kinds of applications is AMT-style programming better suited to creating maintainable code than message passing-style code?
    Are AMT programs more difficult to debug because it is (1) easier to create race conditions and (2) more difficult to get anything like a stack trace? What is being done to make debugging of large-scale AMT applications easier?
    \item
    What sort of language support is needed for AMT programming?
    \begin{itemize}
        \item  What advantage do newer languages like Go, Julia, etc., have for AMT programming, and can/should they displace C++ for scientific applications?
        \item Many languages have added “await” or some variant as new keywords for processing asynchronous code. How important is this feature for the success of the AMT paradigm?
    \end{itemize}
    \item
    How to choose which AMT programming model among others?
\end{itemize}

\section{List of abstracts}



\end{document}
